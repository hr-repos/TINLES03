
\section{Testing Procedures}

\subsection{Driving of the bot}
The robot was tested in a controlled indoor environment on a flat, smooth surface. The following experiments were conducted:

\begin{enumerate}
    \item \textbf{Time per Rotation Test}: The rotation time of each motor was recorded, conducting 
        \textbf{five} and \textbf{ten} consecutive rotation trials per motor to analyze consistency and speed. 
        This is to make sure that the rotation speed is the same across all motors
    \item \textbf{Linear Motion Test}: The robot was commanded to move forward, backward, left, and right, to 
        assess its ability to maintain straight-line movement.
    \item \textbf{Diagonal Motion Test}: The capability of moving diagonally without requiring additional turning was evaluated.
    \item \textbf{Rotational Motion Test}: The robot performed 360-degree in-place rotations to measure its turning efficiency.
    \item \textbf{Complex Maneuverability Test}: The robot followed a predefined movement path incorporating sharp turns and 
        continuous directional changes to assess real-world usability.
\end{enumerate}

\subsection{sensors} 
\textbf{Precision of the ultrasonic sensors}:
The ultrasonic sensors, used to detect obstacles, were tested by placing obstacles at various distances and 
measuring the distance reported by the sensors. The sensors were able to detect obstacles
with a deviation of 2\%.


\textbf{Limitations of the gyroscopic sensor}:
The gyroscopic sensor is limited to measurements on the x/y-axis and is not able to measure the heading/yaw.
Because of the lack of heading and yaw measurements, the gyro is not suitable for our purpose.


\textbf{Performance of the magnetometer sensor}:
The magnetometer sensor was tested by rotating the robot in a circle and measuring the heading.
The sensor was able to detect the heading with a deviation of 2\%. The sensor was able to detect the heading with a deviation of 2\%.

\textbf{Coordinate transformation}:
The calculation was performed using trigonometry. Initially, the rotation was inverted due to the calculations, but after making the necessary adjustments, this issue was successfully resolved.
To verify the accuracy, the rotation angle was taken and recalculated using the obtained values. The results matched, confirming the correctness of the approach.
By utilizing the heading and the correct sensor values, recalculating the rotation is therefore possible and reliable.